\documentclass[../main.tex]{subfiles}

\begin{document}

\subsection{Линейные отображения}
Пусть $V_1, V_2$ - два линейных (конечномерных) пространства над полем F.

\void\stmnt{Опр} Отображение $\varphi: V_1\rightarrow V_2$ называется линейным, если:

\numsec{1} $\forall x,y\in V_1$ $\varphi(x+y) = \varphi(x) + \varphi(y)$,

\numsec{2} $\forall x\in V_1$, $\forall\alpha\in F$ $\varphi(\alpha x) = \alpha\varphi(x)$.

\void\stmnt{Зам} Линейное отображение $\varphi$ - это гомоморфизм линейных пространств, т.е. $\varphi\in Hom(V_1,V_2)$.

\void
Пусть $e_1,...,e_n$ - базис в $V_1$, а $f_1,...,f_m$ - базис в $V_2$ ($dimV_1 = n, dimV_2 = m$).

Рассмотрим $\varphi(e_1),...,\varphi(e_n)\in V_2$ и разложим их по базису $f_1,...,f_m$ в $V_2$:

$$\begin{matrix}
    \varphi(e_1) = a_{11} f_1 + a_{21} f_2 + ... + a_{m1} f_m,\\
    .\\
    .\\
    .\\
    \varphi(e_n) = a_{1n} f_1 + a_{2n} f_2 + ... + a_{mn} f_m.
\end{matrix}$$

\void\stmnt{Опр} Матрица линейного отображения - это матрица

$$A_{ef} = \mx{
    a_{11} & . & . & . & . & a_{1n}\\
    a_{21} & . & . & . & . & a_{2n}\\
    . & . & . & . & . & .\\
    . &  &  &  &  & .\\
    . &  &  &  &  & .\\
    a_{m1} & . & . & . & . & a_{mn}\\
}$$

по столбцам которой стоят координаты образов векторов базиса $V_1$ в базисе $V_2$.

\void\stmnt{Опр} Линейное отображение пространства V в себя называется линейным оператором.
$$f: V\rightarrow V$$

\void\stmnt{Опр} Пусть $e_1,...,e_n$ - базис в $V_1$, а $f: V\rightarrow V$ - линейный оператор (л.о.), тогда матрица

$$A_{e} = \mx{
    a_{11} & . & . & . & . & a_{1n}\\
    a_{21} & . & . & . & . & a_{2n}\\
    . & . & . & . & . & .\\
    . &  &  &  &  & .\\
    . &  &  &  &  & .\\
    a_{n1} & . & . & . & . & a_{nn}\\
}$$

называется матрицей линейного оператора, если

$$\begin{matrix}
    \varphi(e_1) = a_{11} e_1 + a_{21} e_2 + ... + a_{n1} e_n,\\
    .\\
    .\\
    .\\
    \varphi(e_n) = a_{1n} e_1 + a_{2n} e_2 + ... + a_{nn} e_n.
\end{matrix}$$

\void\stmnt{Пример} $V_3$ - трехмерное геометрическое пространство, $L = L(i)$ - подпространство ($i$ - это 1-й вектор из стандартного базиса $i,j,k$).

$$
\begin{matrix}
    f(x) = pr_L^x\\
    f(i) = i = 1\cdot i + 0\cdot j + 0\cdot k\\
    f(j) = 0\\
    f(k) = 0
\end{matrix}
\Longrightarrow A_{\{i,j,k\}} =
\mx{1&0&0\\0&0&0\\0&0&0}
$$

\void\stmnt{Утв} (о том, что линейный оператор полностью задается матрицей при фиксированном базисе)

Пусть $f$ - линейным оператор в пространстве V, $e = \{e_1,...,e_n\}$ - базис в V. Пусть $x\in V$ и
$x^e = (x_1,...,x_n)^T$ - столбец координат вектора $x$ в базисе $e$.

Пусть $A_e$ - матрица линейного оператора $f$ в базисе $e$. Тогда $(f(x))^e = A_e\cdot x^e$.

\void $\square$ 
$$
f(x) = f(x_1e_1 + ... + x_n e_n) = x_1\cdot f(e_1) + ... + x_n\cdot f(e_n) =
x_1(a_{11}e_1 + ... + a_{n1}e_n) + ... + x_n(a_{1n}e_1 + ... + a_{nn}e_n) =$$
$$
(a_{11}x_1 + a_{12}x_2 + ... + a_{1n}x_n)e_1 + ... + (a_{n1}x_1+...+a_{nn}x_n)e_n
$$

$$
\Longrightarrow (f(x))^e =
\mx{
    a_{11}x_1 + a_{12}x_2 + ... + a_{1n}x_n\\
    .\\
    .\\
    .\\
    a_{n1}x_1 + a_{n2}x_2 + ... + a_{nn}x_n
} = A_e\cdot x^e \blacksquare
$$

\void\stmnt{Утв} Пусть $\varphi$ - линейное отображение из пространства $V_1$ в пространство $V_2$.
Пусть $A_{E_1E_2}$ - матрица линейного отображения в паре базисов: $E_1$ - базис в $V_1$, $E_2$ - базис в $V_2$.

Пусть даны две матрица перехода:

$T_1$ - матрица перехода от $E_1$ к $E_1^{'}$ в $V_1$,

$T_2$ - матрица перехода от $E_2$ к $E_2^{'}$ в $V_2$.

Тогда матрица линейного отображения в новой паре базисов $A_{E_1^{'}E_2^{'}} = T_2^{-1}\cdot A_{E_1E_2}\cdot T_1$.

\void
$\square$

$x^{E_1^{'}} = T_1^{-1}\cdot x^{E_1}$ - формула для замены координат в $V_1$,

$y^{E_2^{'}} = T_2^{-1}\cdot y^{E_2}$ - формула для замены координат в $V_2$.

Пусть $y$ - образ $x$ под действием $\varphi$ ($y = \varphi(x)$). Тогда по предыдущему утверждению

$y^{E_2} = A_{E_1E_2}\cdot x^{E_1}$ и $y^{E_2^{'}} = A_{E_1^{'}E_2^{'}}\cdot x^{E_1^{'}}$

$\Longrightarrow T_2^{-1} = y^{E_2} = A_{E_1^{'}E_2^{'}}\cdot T_1^{-1}\cdot x^{E_1} \Longrightarrow
y^{E_2} = \ub{ T_2\cdot A_{E_1^{'}E_2^{'}}\cdot T_1^{-1} }_{= A_{E_1E_2}}\cdot x^{E_1}$

$\Longrightarrow A_{E_1^{'}E_2^{'}} = T_2^{-1}\cdot A_{E_1E_2}\cdot T_1$ $\blacksquare$

\void\stmnt{Следствие} Пусть $\varphi$ - это $f$, то есть линейный оператор. Тогда $E_1 = E_2 = E$, $E_1^{'} = E_2^{'} = E^{'}$.

Тогда формула принимает вид $A_{E^{'}} = T^{-1}\cdot A_E\cdot T$.

\void\stmnt{Утв} С каждым линейным отображением $\varphi: V_1\rightarrow V_2$ связаны два подпространства:

$$Ker(\varphi)\subseteq V_1, Im(\varphi)\subseteq V_2$$

\void\stmnt{Опр} Ядром линейного отображения $\varphi$ называется

$$Ker\varphi = \{x\in V_1\vert \varphi(x) = 0\} = \varphi^{-1}(0)$$

\void\stmnt{Опр} Образом линейного отображения называется

$$Im\varphi = \{ x\in V_2\vert \exists y\in V_1: \varphi(y) = x \} = \varphi(V_1)$$

\void\stmnt{Зам} $Ker\varphi$ и $Im\varphi$ являются подпространствами.

\void\stmnt{Утв} Пусть $\varphi: V_1\rightarrow V_2$. Тогда $dim(Ker\varphi) + dim(Im\varphi) = m = dim(V_1)$.

\void
$\square$ Выберем базис в $V_1$: $e = \{e_1,...,e_m\}$. Тогда любой $x\in V_1$ можно представить в виде:

$$x = x_1e_1 + ... + x_m e_m\Longrightarrow \varphi(x) = x_1\cdot\varphi(e_1) + ... + x_m\cdot\varphi(e_m)$$

$Im\varphi = L(\varphi(e_1),...,\varphi(e_m))\Longrightarrow dim(Im\varphi) = RgA$.

Ядро линейного отображения записывается системой $Ax = 0\Longrightarrow dim(Ker\varphi)$ - это число элементов в ФСР
$Ax = 0$.

Но число элементов в ФСР - это $m - RgA$, то есть $m - RgA = dim(Ker\varphi)\Longrightarrow dim(Ker\varphi) + dim(Im\varphi) = m$.
$\blacksquare$

\void\stmnt{Зам} Пусть $\varphi$ - это линейный оператор, $\Longrightarrow Ker\varphi, Im\varphi$ - подпространства одного
линейного пространства V ($\varphi: V\rightarrow V$). Вообще говоря, $V\neq Ker\varphi\oplus Im f$, где
$\varphi$ - линейный оператор. ($dimV = dim(Ker\varphi) + dim(Im\varphi)$).

\void\stmnt{Пример} $D: g\mapsto g^{'}$ в $\R_n[x]$ (D - операция дифференцирования).

\void
$dim(\R_n[x]) = n+1$,

$ImD = \R_{n-1}[x]\Longrightarrow dim(ImD) = n$,

$KerD = L(1)$, $dim(KerD) = 1$.

$\ub{dim(ImD)}_{n} + \ub{dim(KerD)}_{1} = n+1$.

Но $KerD\cap ImD\neq \{0\}$ и $KerD+ImD = \R_{n-1}[x]\neq \R_n[x]$.

\subsubsection{Действия над линейными отображениями и их матрицами}

\void\stmnt{Опр} Пусть $A, B$ - линейные опраторы на пространстве V (над полем F), $\lambda\in F$, тогда:

$$(A+B)(x) = A(x)+B(x)$$
$$(\lambda A)(x) = \lambda\cdot A(x)$$
$$(A\cdot B)(x) = A(B(x))$$

\void\stmnt{Утв} Если фиксировать базис $e = \{e_1,...,e_n\}$, то

$$(A+B)^e = A^e + B^e$$
$$(\lambda A)^e = \lambda\cdot A^e$$
$$(A\cdot B)^e = A^e\cdot B^e$$

\void $\square$

$$((AB)x)^e = (A(Bx))^e = A_e\cdot(Bx)^e = A_e\cdot B_e\cdot x^e = (A_e\cdot B_e)\cdot x^e$$

$\blacksquare$

\void\stmnt{Опр} Две матрицы А и В $\in M_n(F)$ называются подобными, если $\exists\; C:\; detC\neq 0, A = C^{-1}\cdot B\cdot C$.

\void\stmnt{Зам} Матрицы одного и того же линейного оператора в разных базисах подобны между собой $(A^{'} = T^{-1}\cdot A\cdot T)$.

\void\stmnt{Утв} Определители подобных матриц равны.

\void $\square$ $A = C^{-1}\cdot B\cdot C \Longrightarrow detA = det(C^{-1}\cdot B\cdot C) =
det (C^{-1})\cdot detB\cdot detC = detB\cdot det(\ub{C^{-1}\cdot C}_{E}) = detB$. $\blacksquare$

\void\stmnt{Следствие} $det(A_e)$ - определитель матрицы линейного оператора не зависит от выбора базиса
(т.е. является инвариантом).

\void\stmnt{Опр} Число $\lambda$ называется собственным числом (или собственным значением (сократим до \textbf{с.з.}))
линейного оператора $\varphi: V\rightarrow V$, если существует вектор $x\in V\neq 0$, т.ч. $\varphi(x) = \lambda x$.

При этом вектор $x$ называется собственным вектором, отвечающим собственному значению $\lambda$.

\void\stmnt{Зам} Собственный вектор - это ненулевой вектор, остающийся коллинеарным самому себе
под действием оператора $\varphi$. В данном случае собственное значение - это коэффициент пропорциональности.

\void\stmnt{Пример} Существуют линейные операторы, для которых нет собственных векторов над данным полем.

\void\numsec{1} $V = \R^2$, оператор - поворот на угол $\alpha$.

$$\text{Матрица оператора}\; A = \mx{cos\alpha & -sin\alpha\\ sin\alpha & cos\alpha}$$

Если $\alpha\neq \pi k\;\; (k\in\Z)$, то ни один вектор не переходит в коллинеарный.

\void\numsec{2} $\varphi = id$ в $\R^2$, тогда любой ненулевой вектор - собственный.

\void\stmnt{Опр} Для произвольной квадратной матрицы A определитель $\chi(\lambda) = det(A-\lambda E)$ называют
характеристическим многочленом матрицы А.

\void\stmnt{Утв} Характеристические многочлены подобных матриц совпадают (обратное неверно!).

\void $\square$ Матрицы $A^{'}$ и $A$ подобны $\Leftrightarrow$ $\exists\; T:\; detT\neq 0$, $A^{'} = T^{-1}\cdot A\cdot T$.

$$ \chi_{A^{'}}(\lambda) = det(A^{'} - \lambda E) = det(T^{-1}\cdot A\cdot T - T^{-1}\cdot\lambda\cdot T) =
det(T^{-1}(A-\lambda E) T) = $$
$$= det(T^{-1})\cdot det(A-\lambda E)\cdot detT = det(T^{-1} T)\cdot det(\ub{A-\lambda E}_{\chi_A(\lambda)}) = \chi_A(\lambda)\;\; \blacksquare$$

\void\stmnt{Следствие} Характеристичесские многочлены для матриц линейного оператора в разных базисах
совпадают. Корректно говорить про характеристический многочлен линейного оператора.

\void\stmnt{Опр} Характеристическим уравнением называют уравнение $\chi_A(\lambda) = 0$.

\void\stmnt{Опр} Множество всех собственных значений (для конечномерного оператора) называют \textbf{спектром}
линейного оператора.

\void\stmnt{Теорема} $\lambda$ - собственное значение линейного оператора А $\Leftrightarrow$
$\lambda$ - корень характеристического многочлена (над алгебраически замкнутым полем или если корень
принадлежит рассматриваемому полю $F$).

\void $\square$ \textbf{Необходимость}

\textbf{Дано}: $\lambda\in$ спектру. \textbf{Доказать}: $\lambda$ - корень характеристического уравнения.

$\lambda\in$ спектру $\Leftrightarrow \exists x\neq 0:\;\; Ax = \lambda x$, то есть $Ax = \lambda\cdot I\cdot x$ 
($I$ - тождественный оператор), то есть $(\ub{A-\lambda I}_{\text{новый оператор}})x = 0$ (1)

Запишем равенство (1) в некотором базисе:

$(A_e -\lambda E)\cdot x^e = 0$ ($A_e$ - матрица л.о. в б-се $e$, $x^e$ - столбец коор-т век-ра $x$ в б-се $e$)
- это однородная СЛАУ, которая имеет ненулевое решение $\Longrightarrow det(A_e - \lambda E) = 0$, т.е.
$\lambda$ - корень уравнения.

\void\textbf{Достаточность}

\textbf{Дано}: $\lambda$ - корень характеристического уравнения $\chi_A(\lambda) = 0$.
\textbf{Доказать}: $\lambda$ - собственное значение А.

Если $\lambda$ - корень, то в заданном базисе выполняется равенство $det(A_e-\lambda E) = 0$, следовательно,
соответствующая СЛАУ с матрицей $A_e-\lambda E$ имеет ненулевое решение (выполняется критерий существования
нетривального решения ОСЛАУ с квадратной матрицей). Решение обозначим как $x^e$. Его можно интерпретировать
как набор координат некоторого вектора, для которого выполняется

\void (1) $(A \lambda I)x = 0$, $x\neq 0$, а это по определению означает, что $x$ - собственный
вектор, а $\lambda$ - собственное значение. $\blacksquare$

\void\stmnt{Опр} Алгебраической кратностью собственного значения называется кратность $\lambda$ как
корня характеристического уравнения.

\void\stmnt{Пример} $\chi_A(\lambda) = (\lambda - 5)^3\cdot(\lambda - 7)^2$. Тогда у собственного
значения $\lambda_1 = 5$ алгебраическая кратность равна трем, у $\lambda_2 = 7$ - двум.

\void\stmnt{Утв} Пусть $A: V\rightarrow V$ - линейный оператор  и $\lambda$ - его собственное значение.
Тогда множество $V_{\lambda} = \{ x\in V\;\vert\; A(x) = \lambda x \}$ - подпространство в $V$, называется
собственным, отвечающим $\lambda$.

\void $\square$ $V_{\lambda} = \ker(A - \lambda E)$, а $KerB$, где $B$ - любой линейный оператор,
всегда является подпространством. $\blacksquare$

\void\stmnt{Опр} Размерность подпространства $V_{\lambda}$ называется геометрической кратностью
собственного значения $\lambda$; геометрическая кратность = $dim(V_{\lambda}) = dim(Ker(A-\lambda I))$.

\void\stmnt{Утв} Геометрическая кратность собственного значения (число линейно независимых собственных векторов,
отвечающих $\lambda$) не превышает его алгебраической кратности.

\void\stmnt{Утв} След матрицы линейного оператора не зависит от выбора базиса.

\void $\square$ $TrA$ - это коэффициент при $\lambda^{n-1}$ (с точностью до знака) в характеристическом многочлене,
а весь характеристическим многочлен - это инвариант.

$$\mx{
    a_{11}-\lambda & . & . & . \\
    . & a_{22}-\lambda & . & . \\
    . & . & . & . \\
    . & . & . & a_{nn}-\lambda
} = A - \lambda E\;\; \blacksquare$$

\void\stmnt{Утв} Пусть $\lambda_1,...,\lambda_k$ - попарно различные собственные значения линейного оператора А,
а $v_1,...,v_k$ - соответстсующие собственные векторы. Тогда эти векторы линейно независимы.

\void $\square$ Докажем по индукции. При $k = 1$ - верно, т.к. соответствующий вектор не нулевой и, очевидно,
линейно независим.

Пусть утверждение верно для $k=m$. Добавим еще один собственный вектор $v_{m+1}$. Докажем, что система 
$v_1,...,v_m,v_{m+1}$ осталось линейно независимой.

Рассмотрим равенство $\alpha_{1} v_{1} + \alpha_{2} v_{2} + ... + \alpha_{m} v_{m} + \alpha_{m+1} v_{m+1} = 0$ (2)

\void Применим к (2) линейный оператор A:

$$\alpha_1 \ub{A v_1}_{\text{с.в.}} + ... + \alpha_{m+1} \ub{A v_{m+1}}_{\text{с.в.}} = 0 \Longrightarrow
\alpha_1\lambda_1 v_1 + ... + \alpha_{m+1}\lambda_{m+1} v_{m+1} = 0\;\;\; (3)$$

Умножим (2) на $\lambda_{m+1}$ и вычтем из него (3):

$\alpha_1(\lambda_1 - \lambda_{m+1})v_1 + ... + \alpha_m(\lambda_m - \lambda_{m+1})v_m = 0$, так как все
$\lambda_i$ различны, а $v_1,...,v_m$ линейно независимы, то

$$\left\{
    \begin{matrix}
        \alpha_1(\lambda_1 - \lambda_{m+1}) = 0\\
        . \\
        . \\
        . \\
        \alpha_m(\lambda_m - \lambda_{m+1}) = 0
    \end{matrix}
\right. \Longrightarrow \left\{
    \begin{matrix}
        \alpha_1 = 0\\
        . \\
        . \\
        . \\
        \alpha_m = 0
    \end{matrix}
\right.$$

$\Longrightarrow$ (2) можно записать в виде $\alpha_{m+1}\cdot v_{m+1} = 0$, а так как $v_{m+1}$ -
собственный вектор, то $v_{m+1}\neq 0 \Longrightarrow \alpha_{m+1} = 0\Longrightarrow$ по определению
линейной независимости векторов векторы $v_1,...,v_{m+1}$ действительно линейно независимы. $\blacksquare$

\end{document}